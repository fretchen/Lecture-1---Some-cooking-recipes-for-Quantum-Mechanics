This is the first lecture of the Advanced Atomic Physics course at Heidelberg University, as tought in the wintersemester 2019/2020.
It is intended for master students, which have a basic understanding of quantum mechanics and electromagnetism. In total, we will study multiple topics of modern atomic, molecular and optical physics over a total of roughly 25 lectures, where each lectures is approximately 90 minutes. The topics of the lectures will be discussed in more details in the associated tutorials. 

\section{Principles of Quantum mechanics}
In AMO physics we will encounter the consequences of quantum mechanics all the time. So we will start out with a review of the basic ingredients to facilitate the later discussion of the experiments. Some good introductions can be found in \cite{2002, 2006} \cite{1}\cite{2}. We will mostly follow the discussion of Ref. \cite{2006}.

\subsection{Identify a Suitable Hilbert Space for the Problem in Question}
The first step is to identify the right Hilbert space for your problem. Mathematically, it will be a complex vector space with elements $\ket{\psi}$ and a Hermitian scalar product
\begin{align}
				\langle\psi_1 \psi_2\rangle=(\langle{\psi_2}| \psi_1\rangle)^*.
\end{align}

\textbf{Some examples}:
\begin{enumerate}
\item \textit{Orbit in an atom, molecule etc}. The structure of the Hilbert space will be discussed in more detail for the case of the hydrogen atom
\item \textit{Occupation number of a photon mode}. This Hilbert space is typically investigated in great detail in quantum optics.
\item \textit{Position of an atom} is of great importance for double slit experiments, the quantum simulation of condensed matter systems with atoms,  or matterwave experiments.
\item The \textit{arm of an interferometer} is the canonical way of introducing phase dependent paths and detecting interference patterns.
\item The \textit{spin degree of freedom} of an atom like in the historical Stern-Gerlach experiment. The Hilbert space is now really simple as it reads:
\begin{align}
	\ket{\psi} = a_1 \ket{\uparrow} + a_2 \ket{\downarrow}  \qquad \text{with} \; \langle\psi | \psi\rangle = 1.
\end{align}
\end{enumerate}