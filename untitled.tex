This is the first lecture of the Advanced Atomic Physics 

In AMO physics we will encounter the consequences of quantum mechanics all the time. So we will start out with a review of the basic ingredients to facilitate the later discussion of the experiments. SOME GOOD REFERENCES HERE.

\section{Principles}

\subsection{Identify a Suitable Hilbert Space for the Problem in Question}

In the following, we consider a complex vector space with elements $\ket{\psi}$ and a Hermitian scalar product
\begin{align}
				\langle\psi_1 \psi_2\rangle=(\langle{\psi_2}| \psi_1\rangle)^*.
\end{align}

\textbf{Example 1}: Stern-Gerlach experiment.

Here, we are only interested in the internal degree of freedom of the atoms:

\begin{itemize}
						\item 	We ignore the motion of the 47 electrons and the center of mass motion.
						\item 	For a measurement, we correlate the internal degree of freedom to the spatial degree of freedom. This is done by applying a magnetic field gradient acting on the magnetic moment $\hat{\vec{\mu}}$ \index{magnetic moment}, which in turn is associated with the spin via $\hat{\vec{\mu}} = g \mu_B \hat{\vec{s}}/\hbar$, where $g$ is the Landé $g$-factor \index{Land\'e $g$-factor} and $\mu_B$ is the Bohr magneton \index{Bohr magneton}. The energy of the system is $\hat{H} = -\hat{\vec{\mu}} \cdot \vec{B}$.
						\item 	We consider an isolated system and describe it by a \emph{pure} state vector

\begin{align}
	\ket{\psi} = a_1 \ket{\uparrow} + a_2 \ket{\downarrow}  \qquad \text{with} \; \langle\psi | \psi\rangle = 1.
\end{align}
					
\end{itemize}