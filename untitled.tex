This is the first lecture of the Advanced Atomic Physics course at Heidelberg University in the wintersemester 2018/2019. It is intended for master students, which have a basic understanding of quantum mechanics and electromagnetism. In total, we will study multiple topics of modern atomic, molecular and optical physics over a total of roughly 25 lectures, where each lectures is approximately 90 minutes. The topics of the lectures will be discussed in more details in the associated tutorials. 

\section{Principles of Quantum mechanics}
In AMO physics we will encounter the consequences of quantum mechanics all the time. So we will start out with a review of the basic ingredients to facilitate the later discussion of the experiments. Some good introductions can be found in \cite{2002, 2006} \cite{1}\cite{2}. We will mostly follow the discussion of Ref. \cite{2006}.

\subsection{Identify a Suitable Hilbert Space for the Problem in Question}
The first step is to identify the right Hilbert space for your problem. Mathematically, it will be a complex vector space with elements $\ket{\psi}$ and a Hermitian scalar product
\begin{align}
				\langle\psi_1 \psi_2\rangle=(\langle{\psi_2}| \psi_1\rangle)^*.
\end{align}

\textbf{Example 1}: Stern-Gerlach experiment.

Here, we are only interested in the internal degree of freedom of the silver atoms:

\begin{itemize}
						\item 	We ignore the motion of the 47 electrons and focus on the center of mass motion.
						\item 	For a measurement, we correlate the internal degree of freedom to the spatial degree of freedom. This is done by applying a magnetic field gradient acting on the magnetic moment $\hat{\vec{\mu}}$ \index{magnetic moment}, which in turn is associated with the spin via $\hat{\vec{\mu}} = g \mu_B \hat{\vec{s}}/\hbar$, where $g$ is the Landé $g$-factor \index{Land\'e $g$-factor} and $\mu_B$ is the Bohr magneton \index{Bohr magneton}. The energy of the system is $\hat{H} = -\hat{\vec{\mu}} \cdot \vec{B}$.
						\item 	We consider an isolated system and describe it by a \emph{pure} state vector

\begin{align}
	\ket{\psi} = a_1 \ket{\uparrow} + a_2 \ket{\downarrow}  \qquad \text{with} \; \langle\psi | \psi\rangle = 1.
\end{align}
\end{itemize}
\textbf{Other examples}:
\begin{itemize}
\item Orbit in an atom, molecule etc. The structure of the Hilbert space will be discussed in more detail for the case of the hydrogen atom
\item Occupation number of a photon mode. This Hilbert space is typically investigated in great detail in quantum optics.
\item Position of an atom is of great importance for interferthe quantum simulation of condensed matter systems with atoms, interfer
\item Arm of an interferometer.
\end{itemize}
