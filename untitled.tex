This is the first lecture of the Advanced Atomic Physics course at Heidelberg University, as tought in the wintersemester 2019/2020.
It is intended for master students, which have a basic understanding of quantum mechanics and electromagnetism. In total, we will study multiple topics of modern atomic, molecular and optical physics over a total of 24 lectures, where each lectures is approximately 90 minutes. 
\begin{itemize}
\item We will start the series with some basics on quantum mechanics.
\item Then work our way into the harmonic oscillator and the hydrogen atom.
\item We will then leave the path of increasingly complex atoms for a moment to have some fun with light-propagation, lasers and discussion of the Bell inequalities.
\item A discussion of more complex atoms gives us the acutual tools at hand that are in the lab.
\item This sets up a discussion of di-atomic molecules, which ends the old-school AMO.
\item We move on to quantized atom-light interaction, the Jaynes Cummings model and strong-field lasers.
\item We will finally finish with modern ways to implement quantum simulators and quantum computers.
\end{itemize}

The topics of the lectures will be discussed in more details in the associated tutorials. 

\section{Introduction}
In AMO physics we will encounter the consequences of quantum mechanics all the time. So we will start out with a review of the basic ingredients to facilitate the later discussion of the experiments. 

Some good introductions on the traditional approach can be found in \cite{2002, 2006} \cite{1}\cite{2}. Previously, we mostly followed the discussion of Ref. \cite{2006}. Nowadays, I also recommend the works by Scott Aaronson \cite{quantum,holes}. There is also a good \href{https://www.quantamagazine.org/quantum-theory-rebuilt-from-simple-physical-principles-20170830/#}{article by Quanta-Magazine} on the whole effort to derive quantum mechanics from some simple principles. This effort started with Ref. \cite{axioms}, which actually makes for a nice read.

Before we start with the detailled cooking recipe let us give you some examples of  quantum systems, which are of major importance throughout the lecture:
\begin{enumerate}
\item \textit{Orbit in an atom, molecule etc}. Most of you might have studied this during the introduction into quantum mechanics.
\item \textit{Occupation number of a photon mode}. Any person working on quantum optics has to understand the quantum properties of photons.
\item \textit{Position of an atom} is of great importance for double slit experiments, the quantum simulation of condensed matter systems with atoms,  or matterwave experiments.
\item The \textit{spin degree of freedom} of an atom like in the historical Stern-Gerlach experiment. 
\item The classical coin-toss or bit, which connects us nicely to simple classical probability theory or computing
\end{enumerate}

\section{The possible outcomes  (the Hilbert Space) for the Problem in Question}



The first step is to identify the right Hilbert space for your problem. For a classical problem, we would simply list all the different possible outcomes in a list $(p_1, \cdots, p_N)$. As one of the outcomes has to happen, we obtain the normalization condition:
\begin{align}
\end{align}

Mathematically, it will be a complex vector space with elements $\ket{\psi}$ and a Hermitian scalar product
\begin{align}
				\langle\psi_1 \psi_2\rangle=(\langle{\psi_2}| \psi_1\rangle)^*.
\end{align}

\textbf{Some examples}:
\begin{enumerate}
\item \textit{Orbit in an atom, molecule etc}. The structure of the Hilbert space will be discussed in more detail for the case of the hydrogen atom
\item \textit{Occupation number of a photon mode}. This Hilbert space is typically investigated in great detail in quantum optics.
\item \textit{Position of an atom} is of great importance for double slit experiments, the quantum simulation of condensed matter systems with atoms,  or matterwave experiments.
\item The \textit{arm of an interferometer} is the canonical way of introducing phase dependent paths and detecting interference patterns.
\item The \textit{spin degree of freedom} of an atom like in the historical Stern-Gerlach experiment. The Hilbert space is now really simple as it reads:
\begin{align}
	\ket{\psi} = a_1 \ket{\uparrow} + a_2 \ket{\downarrow}  \qquad \text{with} \; \langle\psi | \psi\rangle = 1.
\end{align}
\end{enumerate}