\subsection{Observables and Measurements}

A given physical quantity $A$ is associated with a Hermitian operator $\hat{A}$ \index{Hermitian operator} acting in the Hilbert space \index{Hilbert space} of the system.
In a \emph{measurement} , the possible outcomes are the eigenvalues $a_\alpha$ \index{eigenvalue} of the operator $\hat{A}$:

\begin{align}
				\hat{A}\ket{\alpha}=a_{\alpha}\ket{\alpha}.
			
\end{align}
The system will collapse to the corresponding eigenvector \index{eigenvector} and the probability of finding the system in state $\ket{\alpha}$ is

\begin{align}
	P(\ket{\alpha})=||\hat{P}_{\ket{\alpha}} \ket{\psi}||^2 = \bra{\psi} \hat{P}^{\dag}_{\ket{\alpha}} \hat{P}_{\ket{\alpha}} \ket{\psi},
\end{align}
where $\hat{P}_{\ket{\alpha}}= \ket{\alpha} \bra{\alpha}$.

As for our previous examples, how would you measure them typically, i.e. what would be the operator ?
\begin{itemize}
\item To measure the rbit we typically perform spectroscopy.
\item For the occupation number we have nowadays number counting photodectors.
\item The position of an atom might be detected through high-resolution CCD cameras.CCD camera
\item For the arm of the interferometer we will typically measure the output through photodetectors...
\item For the \textit{measurement of the spin}, we typically correlate the internal degree of freedom to the spatial degree of freedom. This is done by applying a magnetic field gradient acting on the magnetic moment $\hat{\vec{\mu}}$ \index{magnetic moment}, which in turn is associated with the spin via $\hat{\vec{\mu}} = g \mu_B \hat{\vec{s}}/\hbar$, where $g$ is the Landé $g$-factor \index{Land\'e $g$-factor} and $\mu_B$ is the Bohr magneton \index{Bohr magneton}. The energy of the system is $\hat{H} = -\hat{\vec{\mu}} \cdot \vec{B}$.
\end{itemize}