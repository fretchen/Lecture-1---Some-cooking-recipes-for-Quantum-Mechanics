\section{Time Evolution}
Being able to access the operator values and intialize the wavefunction in some way, we also want to have a prediction on its time-evolution. For most cases of this lecture we can simply describe the system by the non-relativistic \textbf{Schrödinger Equation.} It reads
\begin{align}
i\hbar\partial_t\ket{\psi(t)}=\hat{H}(t)\ket{\psi(t)}.
\end{align}
In general, the Hamilton operator $\hat{H}$ is time-dependent. For a time-independent Hamilton operator $\hat{H}$, we can find eigenstates $\ket{\phi_n}$ with corresponding eigenenergies $E_n$ \index{eigenenergy}:

\begin{align}
\hat{H}\ket{\phi_n}=E_n\ket{\phi_n}.
\end{align}
The eigenstates $\ket{\phi_n}$ in turn have a simple time evolution:

\begin{align}
    \ket{\phi_n(t)}=\ket{\phi_n(0)}\cdot \exp{-i E_nt/\hbar}.
\end{align}
If we know the initial state of a system
\begin{align}
\ket{\psi(0)}=\sum_n \alpha_n\ket{\phi_n},
\end{align}
where $\alpha_n=\langle\phi_n | \psi(0)\rangle$, we will know the full dimension time evolution
\begin{align}
\ket{\psi(t)}=\sum_n\alpha_n\ket{\phi_n}\exp{-i E_n t/\hbar}. \;\, \text{(Schrödinger picture)}
\end{align}
\textbf{Note.} Sometimes it is beneficial to work in the Heisenberg picture, which works with static ket vectors $\ket{\psi}^{(H)}$ and incorporates the time evolution in the operators. \footnote{We will follow this route in the discussion of the two-level system and the Bloch sphere.}
In certain cases one would have to have access to relativistic dynamics, which are then described by the \textbf{Dirac equation}. However, we will only touch on this topic very briefly, as it directly leads us into the intruiging problems of \textbf{quantum electrodynamics}.

\subsection{The Heisenberg picture}
As mentionned in the first lecture it can benefitial to work in the Heisenberg picture instead of the Schrödinger picture. This approach is widely used in the field of many-body physics, as it underlies the formalism of the second quantization. To make the connection with  the Schrödinger picture we should remember that we have the formal solution of
\begin{align}
\ket{\psi(t)} = \eexp{-i\hat{H}t}\ket{\psi(0)}
\end{align}
So, if we would like to look into the expectation value of some operator, we have:
\begin{align}
\langle\hat{A}(t)\rangle = \bra{\psi(0)}\eexp{i\hat{H}t}\hat{A}_S\eexp{-i\hat{H}t}\ket{\psi(0)}
\end{align}
This motivates the following definition of the operator in the Heisenberg picture:
\begin{align}
    \hat{A}_H=\eexp{i{\hat{H} t}/{\hbar}} \hat{A}_S \eexp{-i{\hat{H} t}/{\hbar}}
\end{align}
where $\exp{-i{\hat{H} t}/{\hbar}}$ is a time evolution operator (N.B.: $\hat{H}_S = \hat{H}_H$). The time evolution of $\hat{A}_H$ is:
\begin{align}
    \notag \frac{d}{dt} \hat{A}_H &=&& \frac{i}{\hbar}\hat{H}\eexp{i{\hat{H}t}/{\hbar}}\hat{A}_S \eexp{-i{\hat{H} t}/{\hbar}}\\ 
    &&-&\frac{i}{\hbar} \eexp{i{\hat{H} t}/{\hbar}}\hat{A}_S \eexp{-i{\hat{H}t}/{\hbar}}\hat{H}+\partial_t \hat{A}_H\\
    &=&& \frac{i}{\hbar}\left[\hat{H},\hat{A}_H\right] + \eexp{i{\hat{H}t}/{\hbar}}\partial_t\hat{A}_S\eexp{-i{\hat{H}t}/{\hbar}}
\end{align}
\textbf{Note.} In the Heisenberg picture the state vectors are time-independent:
\begin{align}
    \ket{\psi}_H \equiv \ket{\psi(t=0)}=\exp{i{\hat{H}}t/{\hbar}} \ket{\psi(t)}.
\end{align}
Therefore, the results of measurements are the same in both pictures:
\begin{align}
 \bra{\psi(t)}\hat{A}\ket{\psi(t)} = \bra{\psi}_H \hat{A}_H \ket{\psi}_H.
\end{align}