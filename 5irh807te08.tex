\subsection{Time Evolution}
Being able to access the 
\textbf{Schrödinger Equation.} The Schrödinger equation reads
\begin{align}
i\hbar\partial_t\ket{\psi(t)}=\hat{H}(t)\ket{\psi(t)}.
\end{align}
In general, the Hamilton operator $\hat{H}$ is time-dependent. For a time-independent Hamilton operator $\hat{H}$, we can find eigenstates $\ket{\phi_n}$ with corresponding eigenenergies $E_n$ \index{eigenenergy}:

\begin{align}
\hat{H}\ket{\phi_n}=E_n\ket{\phi_n}.
\end{align}
The eigenstates $\ket{\phi_n}$ in turn have a simple time evolution:

\begin{align}
    \ket{\phi_n(t)}=\ket{\phi_n(0)}\cdot \exp{-i E_nt/\hbar}.
\end{align}
If we know the initial state of a system
\begin{align}
\ket{\psi(0)}=\sum_n \alpha_n\ket{\phi_n},
\end{align}
where $\alpha_n=\langle\phi_n | \psi(0)\rangle$, we will know the full dimension time evolution
\begin{align}
\ket{\psi(t)}=\sum_n\alpha_n\ket{\phi_n}\exp{-i E_n t/\hbar}. \;\, \text{(Schrödinger picture)}
\end{align}
\textbf{Note.} Sometimes it is beneficial to work in the Heisenberg picture, which works with static ket vectors $\ket{\psi}^{(H)}$ and incorporates the time evolution in the operators. \footnote{We will follow this route in the discussion of the two-level system and the Bloch sphere.}

