\subsection{Time Evolution}

\textbf{Schrödinger Equation.} The Schrödinger equation reads
\begin{align}
i\hbar\partial_t\ket{\psi(t)}=\hat{H}(t)\ket{\psi(t)}.
\end{align}
In general, the Hamilton operator $\hat{H}$ is time-dependent. For a time-independent Hamilton operator $\hat{H}$, we can find eigenstates $\ket{\phi_n}$ with corresponding eigenenergies $E_n$ \index{eigenenergy}:

\begin{align}
\hat{H}\ket{\phi_n}=E_n\ket{\phi_n}.
\end{align}
The eigenstates $\ket{\phi_n}$ in turn have a simple time evolution:

\begin{align}
    \ket{\phi_n(t)}=\ket{\phi_n(0)}\cdot \exp{-i E_nt/\hbar}.
\end{align}
If we know the initial state of a system
\begin{align}
\ket{\psi(0)}=\sum_n \alpha_n\ket{\phi_n},
\end{align}
where $\alpha_n=\langle\phi_n | \psi(0)\rangle$, we will know the full dimension time evolution
\begin{align}
\ket{\psi(t)}=\sum_n\alpha_n\ket{\phi_n}\exp{-i E_n t/\hbar}. \;\, \text{(Schrödinger picture)}
\end{align}
\textbf{Note.} Sometimes it is beneficial to work in the Heisenberg picture, which works with static ket vectors $\ket{\psi}^{(H)}$ and incorporates the time evolution in the operators. \footnote{We will follow this route in the discussion of the two-level system and the Bloch sphere.}

\subsection{Complete Set of Commuting Observables}

A set of commuting operators $\{\hat{A},\hat{B},\hat{C},\cdots,\hat{X}\}$ is considered a complete set if their common eigenbasis is 				unique. Thus, the measurement of all quantities $\{A,B,\cdots,X\}$ will determine the system uniquely.

\subsection{Entangled States}
 
			Consider a quantum system $S$ formed by two subsystems $S_1$ and $S_2$. For each of them we can write:

\begin{align}
				\ket{\psi_1}&=\sum_m^M a_m \ket{\alpha_m},\\
				\ket{\psi_2}&=\sum_n^N b_n \ket{\beta_n}.
			
\end{align}
			The question is now if it is always possible to write $\ket{\psi}$ in the form

\begin{align}
 \label{eq:psientangled} 
	\ket{\psi}	&=\ket{\psi_1}\otimes\ket{\psi_2}\\
							&=\left(\sum_m^M a_m \ket{\alpha_m}\right) \otimes \left(\sum_n^N b_n \ket{\beta_n}\right) \label{eq:psientangled2} \\
							&=\sum_m^M \sum_n^N a_m b_n \ket{\alpha_m} \otimes \ket{\beta_n}. \label{eq:psientangled3} 
			
\end{align}
\begin{align}
 \label{eq:psientangled} 
	\ket{\psi}	&=\ket{\psi_1}\otimes\ket{\psi_2}\\
							&=\left(\sum_m^M a_m \ket{\alpha_m}\right) \otimes \left(\sum_n^N b_n \ket{\beta_n}\right) \label{eq:psientangled2} \\
							&=\sum_m^M \sum_n^N a_m b_n \ket{\alpha_m} \otimes \ket{\beta_n}. \label{eq:psientangled3} 
			
\end{align}
						From \eqref{eq:psientangled3}, we see that the state $\ket{\psi}$ is determined by $M + N$ numbers. $\ket{\psi}$ can also have the form of a superposition

\begin{align}
\ket{\psi}=\sum_m^M \sum_n^N c_{mn}\ket{\alpha_m}\otimes \ket{\beta_n}.
\end{align}
Other than in \eqref{eq:psientangled3}, $\ket{\psi}$ is determined by $M \times N$ numbers $c_{mn}$ here. If we can not write $\ket{\psi}$ as in \eqref{eq:psientangled}, we call the state \emph{entangled} . Most states are entangled states. Equation \eqref{eq:psientangled2} thus only describes a small subset of all possible states.

\textbf{Example 1: Two Spins.}\label{sec:examplespin}  For a system of two spins we can construct the superposition state

\begin{align}
						\frac{1}{\sqrt{2}}(\ket{\uparrow}\otimes\ket{\downarrow}-\ket{\downarrow}\otimes\ket{\uparrow}).
					
\end{align}
	The minus sign was arbitrarily chosen.
					Note that the first spin has a Hilbert space different from the Hilbert space of the second spin!
				\textbf{Example 2: Electron in a hydrogen atom.} We consider an electron orbiting around a proton. With the tensor product, we can connect the internal degree of freedom of the electron---given by its possible spin states---to its motional degree of freedom, which is characterized by the orbital quantum number $l$ and the magnetic quantum number $m_l$:

\begin{align}
						\left\{

\begin{array}
{c} \ket{\uparrow} \\ \ket{\downarrow} \end{array}
 \right\} \otimes \left\{

\begin{array}
{c} \ket{l=1,m_l=-1} \\ \ket{l=1,m_l=\enspace \; 0} \\ \ket{l=1,m_l=+1}  \end{array}
 \right\}
					
\end{align}
										Unlike \hyperref[sec:examplespin]{Example 1}, it is obvious here that the Hilbert spaces differ from each other.