\section{Quantum rules}
Having set up the space on which we want to act we have to follow the rules of quantum mechanics. The informal way of describing is actually nicely described by Chris Monroe \href{https://youtu.be/CC7nlBM2cSM}{in this video}. We might summarize them as follows:
\begin{enumerate}
\item Quantum objects can be in several states at the same time.
\item Rule number one only works when you are not looking.
\end{enumerate}

The more methematical fashion is two say that there two ways of manipulating quantum states:
\begin{enumerate}
\item Unitary transformations $\hat{U}$.
\item Measurements.
\end{enumerate}

\subsection{Unitary transformations}
As states change and evolve, we know that the total probability should be conserved. So we transform the state by some operator $\hat{U}$, we obtain the condition:
\begin{align}
\bra{\psi}\hat{U}^\dag\hat{U} \ket{\psi} = 1\\
\hat{U}^\dag\hat{U}  = \mathbb{1}
\end{align}
That's the very definition of unitary operators and unitary matrices.
Going back to the case of a coin toss, we see that we can then transform our qubit through the unitary operator:
\begin{align}
\hat{U}=\frac{1}{\sqrt{2}}\left(\begin{array}{cc}
1 & -1\\
1 & 1
\end{array}\right)
\end{align}
Applying it on the previously defined states $\uparrow$ and $\downarrow$, we get the superposition state:
\begin{align}
\hat{U}\ket{\uparrow} &= \frac{\ket{\uparrow}-\ket{\downarrow}}{\sqrt{2}}\\
\hat{U}\ket{\downarrow} &= \frac{\ket{\uparrow}+\ket{\downarrow}}{\sqrt{2}}
\end{align}

Such an operation would not be possible in the classical case, as non-negative values are forbidden there. Actually, operations on classical propability distributions are only possible if every entry of the matrix is non-negative (probabilities are never negative right ?) and each column adds up to one (we cannot loose something in a transformation).