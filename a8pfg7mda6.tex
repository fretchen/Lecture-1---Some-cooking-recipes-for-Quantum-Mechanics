\section{Composite systems}
\subsection{Complete Set of Commuting Observables}

A set of commuting operators $\{\hat{A},\hat{B},\hat{C},\cdots,\hat{X}\}$ is considered a complete set if their common eigenbasis is unique. Thus, the measurement of all quantities $\{A,B,\cdots,X\}$ will determine the system uniquely. The clean identification of such a Hilbert space can be quite challenging and a nice way of its measurment even more. Coming back to our previous examples:

\begin{enumerate}
\item Performing the full spectroscopy of the atom. Even for the hydrogen atom we will see that the full answer can be rather involved...
\item The occupation number is rather straight forward. However, we have to be careful that we really collect a substantial amount of the photons etc.
\item Are we able to measure the full position information ? What is the resolution of the detector and the point-spread function ?
\item Here it is again rather clean to put a very efficient detector at the output of the two arms ...
\item What are the components of the spin that we can access ? The $z$ component does not commute with the other components, so what should we measure ?
\end{enumerate}
\subsection{Entangled States}
 
			Consider a quantum system $S$ formed by two subsystems $S_1$ and $S_2$. For each of them we can write:

\begin{align}
				\ket{\psi_1}&=\sum_m^M a_m \ket{\alpha_m},\\
				\ket{\psi_2}&=\sum_n^N b_n \ket{\beta_n}.
			
\end{align}
			The question is now if it is always possible to write $\ket{\psi}$ in the form

\begin{align}
 \label{eq:psientangled} 
	\ket{\psi}	&=\ket{\psi_1}\otimes\ket{\psi_2}=\left(\sum_m^M a_m \ket{\alpha_m}\right) \otimes \left(\sum_n^N b_n \ket{\beta_n}\right)
\end{align}
\begin{align}
\ket{\psi}	&=\sum_m^M \sum_n^N a_m b_n \ket{\alpha_m} \otimes \ket{\beta_n}. \label{eq:psientangled3} 
\end{align}
From \eqref{eq:psientangled3}, we see that the state $\ket{\psi}$ is determined by $M + N$ numbers. $\ket{\psi}$ can also have the form of a superposition

\begin{align}
\ket{\psi}=\sum_m^M \sum_n^N c_{mn}\ket{\alpha_m}\otimes \ket{\beta_n}.
\end{align}
Other than in \eqref{eq:psientangled3}, $\ket{\psi}$ is determined by $M \times N$ numbers $c_{mn}$ here. If we can not write $\ket{\psi}$ as in \eqref{eq:psientangled}, we call the state \emph{entangled} . Most states are entangled states. Equation \eqref{eq:psientangled} thus only describes a small subset of all possible states.

\textbf{Example 1: Two Spins.}\label{sec:examplespin}  For a system of two spins we can construct the superposition state

\begin{align}
						\frac{1}{\sqrt{2}}(\ket{\uparrow}\otimes\ket{\downarrow}-\ket{\downarrow}\otimes\ket{\uparrow}).
					
\end{align}
	The minus sign was arbitrarily chosen.
					Note that the first spin has a Hilbert space different from the Hilbert space of the second spin!
				\textbf{Example 2: Electron in a hydrogen atom.} We consider an electron orbiting around a proton. With the tensor product, we can connect the internal degree of freedom of the electron---given by its possible spin states---to its motional degree of freedom, which is characterized by the orbital quantum number $l$ and the magnetic quantum number $m_l$:

\begin{align}
						\left\{

\begin{array}
{c} \ket{\uparrow} \\ \ket{\downarrow} \end{array}
 \right\} \otimes \left\{

\begin{array}
{c} \ket{l=1,m_l=-1} \\ \ket{l=1,m_l=\enspace \; 0} \\ \ket{l=1,m_l=+1}  \end{array}
 \right\}
					
\end{align}
Unlike \hyperref[sec:examplespin]{Example 1}, it is obvious here that the Hilbert spaces differ from each other.