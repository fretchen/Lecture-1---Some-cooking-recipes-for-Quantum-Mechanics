\section{Composite systems}
It is actually quite rare that we can label the system with a single quantum number. Any atom will involve spin, position, angular momentum. Other examples might just involve two spin which we observe. So the question is then on how we label those systems. We then have two questions to answer:
\begin{enumerate}
\item How many labels do we need for a system to fully determine its quantum state ?
\item Once I know all the labels, how do I construct the full state out of them ?
\end{enumerate}
We will actually discuss the second question first as it sets the notation for the first question.



\subsection{Entangled States}
 
 Consider now an atomm, which 
 
 
 
 
 
 
 
 
 Consider a quantum system $S$ formed by two subsystems $S_1$ and $S_2$. For two independent
 For each of them we can write:

\begin{align}
				\ket{\psi_1}&=\sum_m^M a_m \ket{\alpha_m},\\
				\ket{\psi_2}&=\sum_n^N b_n \ket{\beta_n}.
			
\end{align}
			The question is now if it is always possible to write $\ket{\psi}$ in the form

\begin{align}
 \label{eq:psientangled} 
	\ket{\psi}	&=\ket{\psi_1}\otimes\ket{\psi_2}=\left(\sum_m^M a_m \ket{\alpha_m}\right) \otimes \left(\sum_n^N b_n \ket{\beta_n}\right)
\end{align}
\begin{align}
\ket{\psi}	&=\sum_m^M \sum_n^N a_m b_n \ket{\alpha_m} \otimes \ket{\beta_n}. \label{eq:psientangled3} 
\end{align}
From \eqref{eq:psientangled3}, we see that the state $\ket{\psi}$ is determined by $M + N$ numbers. $\ket{\psi}$ can also have the form of a superposition

\begin{align}
\ket{\psi}=\sum_m^M \sum_n^N c_{mn}\ket{\alpha_m}\otimes \ket{\beta_n}.
\end{align}
Other than in \eqref{eq:psientangled3}, $\ket{\psi}$ is determined by $M \times N$ numbers $c_{mn}$ here. If we can not write $\ket{\psi}$ as in \eqref{eq:psientangled}, we call the state \emph{entangled} . Most states are entangled states. Equation \eqref{eq:psientangled} thus only describes a small subset of all possible states.



\section{Statistical Mixtures and Density Operator}

If only the subsystem of a pure quantum state is accessible to measurements, or the state of the system is not known at the microscopic level (statistical ensemble!), the state of the system has to be described by a Hermitian density operator

\begin{align}
 \hat{\rho} = \sum_{n=1}^N p_n \ket{\phi_n}\bra{\phi_n}.
\end{align}
Here, $\bra{\phi_n}$ are the eigenstates of $\hat{\rho}$, and $p_n$ are the probabilities to find the system in the respective states $\ket{\phi_n}$.

Let us now have a brief look at the properties of the density operator:
\begin{itemize}
    \item The trace of the density operator is the sum of all probabilities $p_n$:
	%
	\begin{align}
	    \tr{\rhohat} = \sum p_n = 1.
	\end{align}
				%
	\item For a pure state $\ket{\psi}$, we get $p_n=1$ for only one value of $n$. For every other $n$, the probabilities vanish. We thus obtain a ``pure'' density operator $\rhohat_{\text{pure}}$ which has the properties of a projection operator:
				%
	\begin{align}
					\rhohat_{\text{pure}} = \ket{\psi}\bra{\psi} \qquad \Longleftrightarrow \qquad \rhohat^2 = \rhohat.
				\end{align}
\end{itemize}
			%
With this knowledge we can now determine the result of a measurement of an observable $A$ belonging to an operator $\hat{A}$. For the pure state $\ket{\psi}$ we get:
%
\begin{align}
				\langle \hat{A}\rangle = \bra{\psi} \hat{A} \ket{\psi}.
\end{align}
			%
For a mixed state we get:
			%
\begin{align}
	\langle \hat{A}\rangle = \tr{\rhohat \cdot \hat{A}} = \sum_n {p_n} \bra{\phi_n} \hat{A} \ket{\phi_n}.
\end{align}
The time evolution of the density operator can be expressed with the von Neumann equation:
\begin{align}
	i\hbar \partial_{t}\rhohat(t) = [\hat{H}(t),\rhohat(t)].
\end{align}
\textbf{Note.} 
If we consider a system $S = S_1 \otimes S_2$ comprised of two subsystems $S_1$, $S_2$, then the density operator $\hat{\rho}_i$ of  subsystem $i$ is
					%
					\begin{align}
						\rhohat_1=&\trarb{2}{\rhohat},\\
						\rhohat_2=&\trarb{1}{\rhohat},
					\end{align}
					%
					where $\hat{\rho}=\ket{\psi}\bra{\psi}$ and $\trarb{j}{\rhohat}$ is the trace over the Hilbert space of subsystem $j$.


