\subsection{Complete Set of Commuting Observables}

A set of commuting operators $\{\hat{A},\hat{B},\hat{C},\cdots,\hat{X}\}$ is considered a complete set if their common eigenbasis is unique. Thus, the measurement of all quantities $\{A,B,\cdots,X\}$ will determine the system uniquely. The clean identification of such a Hilbert space can be quite challenging and a nice way of its measurment even more. Some possibly rather clean examples:

\begin{enumerate}
\item Performing the full spectroscopy of the atom. Even for the hydrogen atom we will see that the full answer can be rather involved...
\item The occupation number is rather straight forward. However, we have to be careful that we really collect a substantial amount of the photons etc.
\item Are we able to measure the full position information ? Is the detection of the position of the atom really ? What is the resolution of the detector and the point-spread function 
\end{enumerate}

A rather clean example seems to be the detection of the spin in the 